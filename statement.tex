%%%%%% Stuff to add:
% invited (declined) seminars
% extending knowledge: reporters, master gardner, etc.


\documentclass[letterpaper,10pt]{article}

\usepackage{hyperref}
\usepackage{geometry}
\usepackage{etaremune}
\usepackage{verbatim}
%\usepackage{natbib}
\usepackage[T1]{fontenc}
\usepackage[sc,osf]{mathpazo}
\def\name{Jeffrey Ross-Ibarra}
% The following metadata will show up in the PDF properties
\hypersetup{
  colorlinks = true,
  urlcolor = blue,
  pdfauthor = {\name},
  pdfkeywords = {population genetics, maize, plant evolution},
  pdftitle = {\name: Curriculum Vitae},
  pdfsubject = {Curriculum Vitae},
  pdfpagemode = UseNone
}

\geometry{
  body={6.5in, 9in},
  left=1.0in,
  top=1.0in
}

% Customize page headers
\pagestyle{myheadings}
\markright{\name}
\thispagestyle{empty}

% Custom section fonts
\usepackage{sectsty}
\sectionfont{\rmfamily\mdseries\Large}
\subsectionfont{\rmfamily\mdseries\itshape\large}


\begin{document}

% Place name at left
{\huge \name\ -- Candidate Statement}

\vspace{0.25in}

\section*{Summary}

This merit covers the three years from 2015-2018. During this time my research program on the evolutionary genetics of maize and its wild relatives has continued apace.
We have published 23 journal articles (12 from first authors in my lab), been cited nearly 3000 times, and brought in $4.5M$ in new grants.
I completely redesigned my large-lecture undergraduate genetics class and my graduate ecological genetics course.
And I have continued to serve as Section Chair in the department (approximately 10\% time commitment) and have added additional service including chairing the search committee to hire the new department Chair.

\section*{Research}

My lab investigates plant evolutionary genetics, with a focus on maize and its wild relatives.
Broadly, our work can be divided into three areas: breeding and experimental evolution, local adaptation, and genome evolution.
A full list of publications during this period is listed elsewhere; additional details can be found on \href{https://scholar.google.com/citations?user=5SzRq1oAAAAJ&hl=en&authuser=1}{Google Scholar}.

\subsection*{Breeding and Experimental Evolution}
Plant domestication and modern breeding represent examples of experimentally evolved populations.
Studying these populations provides an opportunity to understand not only the genetic basis of evolutionary change but also how the processes of evolution interact to shape modern genetic and phenotypic diversity.

Stemming from our continuing \$3.2M NSF grant (of more than \$13M total) to work on the biology of rare and deleterious alleles, our major contribution during this period has been to show how deleterious mutations may largely explain the phenomenon of hybrid vigor.
The fact that hybrid plants outperform their parents was observed by a number of researchers in the last 19th century, including Darwin.
The genetic basis of the phenomenon is still contentious, however.
One of the simples explanations has been that inbred lines will differ in their complement of deleterious alleles, and these alleles are effectively masked in the hybrid because of dominance.
In two papers during this period, we show that modern breeding strategies facilitate the build up of deleterious alleles differing between pools of inbreds \cite{gerke2015genomic}, and that we can use bioinformatic approaches to \textit{a priori} identify these mutations and use them to predict yield in maize hybrids \cite{yang2017incomplete}.

We have also investigated how social barriers can impact gene exchange \cite{orozco2016maize} and how ancient DNA can inform us about the timing of adaptation during domestication \cite{ramos2016genome}.
In addition to this work we have written two perspectives focusing on gene interaction \cite{Stitzer2018} and the multigenic nature of domestication \cite{stetter2018genetic}.

\subsection*{Local Adaptation}
Maize spread rapidly after domestication, adapting to a wide range of environments.
Today maize is grown across a broader geographic breadth than any of the world's other staple crops \cite{hake2015genetic}, from sea level to altitudes of $>4,000$m and from deserts to near-flooded conditions.
The wild relatives of maize have also adapted to environments varying widely in elevation, temperature, and moisture availability.
During this period we received two grants focusing on adaptation, including a \$4.1M to my lab to study adaptation to high elevation in maize and a \$300K grant as Co-PI to understand loci preventing gene flow between maize and its wild relatives.
Work on local adaptation during this period has looked at phenotypic plasticity \cite{lorant2017potential} and gene expression during adaptation \cite{aguilar2017allele}, the theory of adaptive species interactions and how to test for them \cite{o2015evolutionary}, how maize adapted to the US Southwest \cite{swarts2017genomic}, and how genome size may be an adaptive phenotype \cite{bilinski2018parallel}.
We have we also developed a software package for researchers to make better use of state-of-the-art population genetic approaches to studying adaptation \cite{durvasula2016angsd}.
Finally, and perhaps most exicting for me, we published a novel hypothesis that genome size may determine how plants adapt to new challenges \cite{mei2018adaptation}.
This hypothesis, which we've termed ``functional space hypothesis'' helps explain contrasting results from genomics and population and quantitative genetics between species like maize and the model plant Arabidopsis.
Essentially we argue that the size of large plant genomes determines to a degree how adaptation proceeds --- how many regions of the genome contribute, what kinds of regions they are, etc.
The hypothesis is built on solid population genetic theory and backed up by analysese of existing data sets.
I am optimistic that this new perspective on adaptation in plants will have long-term impacts on how researchers study evolution and identify functional diversity in large plant genomes, including economically important specie like pine, wheat, or corn.

\subsection*{Genome Evolution}
In addition to discerning the genetic basis of phenotypic evolution, we are interested in  understanding the processes that shape evolution of the genome itself.
During this period we investigated the relative importance of different ancestral genomes in contributing to phenotypic variation \cite{renny2017gene}, contributed the first complete transposal element annotation of a plant genome \cite{jiao2017improved}, investigated how selection on one region of the genome can affect diversity at others \cite{beissinger2016recent}, contributed to the genetic analysis of a locus that violates Mendel's rules of segregation \cite{dawe2018kinesin}, provided genomic resources for diversity \cite{bukowski2017construction} and genome assembly \cite{jiao2017improved,wolfgruber2016high}, and tested the utility of methods for estimating repeat abundance \cite{bilinski2017genomic}.

\subsection*{Extending Knowledge}

I have given 17 invited seminars or keynote speeches in four countries during this period, including invitations to speak at academic departments, research institutes, and biotech and breeding companies.
I have also given public presentations to the UC Master Gardeners and the Davis Science Cafe on genetic diversity in maize, the American Society of Plant Biology TapRoot podcast on authorship and work/life balance (my episode has been downloaded more than 2,000 times), and on general genetics to K-12 classrooms in four U.S. states and two classrooms in Spain via the Skype-A-Scientist program.

Finally, I maintain a twitter account that has now accrued more than 5,000 followers, and my science-related tweets (sharing articles, discussing our science) are regularly seen by thousands of people on social media.

\section*{Instruction and Advising}

\subsection*{Instruction}
During this review I taught undergraduate genetics(BIS101) and a graduate course on ecological genetics (ECL243).

ECL243 is the required core course for Ecology graduate students wishing to specialize in ecological genetics.
Following feedback from students we redesigned the course twice during this review.
Student reviews from our most recent iteration suggest we have finally found the right combination of material for the diverse set of students that take the course.
We include both journal article discussion to cover the breadth of ecological genomics as well as a hands-on group project in which students are asked to replicate a computationaly analysis from a published paper of their choosing.
The course is challenging, as many students come in with essentially no computational experience, but the fact that a number of students leave the course planning to change their dissertation to include more genomic analyses suggests it is effective.

BIS101 is a large lecture undergraduate genetics course. I received teaching release for this course in 2015 (from a university award in recognition of service) and 2017 (paternity leave).
I took advantage of the time off in 2015, however, and completely redesigned the course.
I removed a textbook and instead put together free online sources to cover all of the material.
I added online quizzes, incorporated group clicker questions into my lectures, and moved from the chalkboard to all powerpoint presentations.
I added the discussion forum Piazza, allowing studnets to ask anonymous questions and answer each other with feedback from the instructor and TAs.
I also formalized the format and structure of my previous efforts to include primary literature in the course.
Student evaluations increased as a result of these efforts, and one of my top students was sufficiently inspired by these approaches she is now TA for the course this Fall.

\subsection*{Advising}
During this review period three of my PhD students have graduated (two from UC Davis, one from Universit\`{e} Paris-Sud).
One is now a professor at a community college, and two are postdoctoral scholars.
My current graduate students have succesfully found funding from NSF, ARCs, and departmental GSRs; all are progressing on track to graduation.
Six of our publications during this period include graduate students as first author.

I continue to advise the graduate-student-run Plant Science Symposium.
Each year we meet several dozen times to organize a symposium that in recent years has had more than 300 attendees and an industry-funded budget of nearly \$15,000.
I also supervised a US-Mexico student exchange program as part of my NSF grant.

I advise a number of undergraduate students, including several who are now in graduate school researching topics related to evolutionary genetics.
One of the papers pubished during this period includes an undergraduate as first author.

During the review period two postdoctoral scholars succesfully found jobs in industry (Harris Moran, Corteva) and two as Assistant Professors (University of Nebraska, Michigan State). I am currently advising 6 postdoctoral scholars.  Five of the publications from this period include a postdoc as first author.

\section*{Service}

\subsection*{University}
During this review I have continued my efforts as Section Chair of Agricultural Plant Biology. In addition to service on two additional committees, this role requires me to serve as acting chair on occasion, represent the department in meetings with stakeholders or university administrators, write merit and promotion letters for 28 faculty, and act as a backup for the chair in signing HR, accounting, and other forms. In terms of hours invested each year, I estimate in this is equivalent to an almost 10\% appointment.

In addition to this work as Seciton Chair, I took on several new leadership roles in service during this time.  I chaired the search committee for a new department chair.  This was a bit more involved that a normal search, as it involved coordination with the Dean's office and quite a bit of time contacting potential candidates to recruit them to the position. At the University level, I was asked to serve by the former vice provost as a member of the campus disciplinary peer review committee on sexual violence and sexual harassment, which provides  guidance to the vice provost's office on disciplinary measures for senate faculty after an investigation.

I have also continued service to the department and university in a number of additional avenues as detailed in my dossier.

\subsection*{Professional}

During this review period I have continued my role as reivewer and associate editor, as well as serving on a number of committees for societies or journals.

Most importantly, however, I have taken on new leadership roles as a Senior Editor for the journals G3 and PeerJ.
For PeerJ --- a new open-access journal competing with PLoS ONE --- I am one of two Senior Editors overseeing all of plant biology.
My role is to approve decisions from associate editors.  I have overseen almost 50 papers so far, and have stepped in to change or modify decisions a number of times.
G3 is a the sister journal to Genetics from the Genetics Society of America. My role as Senior Editor for population genetics here is more substantial. I help identify and recruit new associate editors, approve editorial decisions (on more than 110 manuscripts so far), and have helped change journal and society policies. I designed and am piloting a new program at the journal to recruit early career investigators as guest associate editors, working with them through every step of the editorial process. This helps the journal recruit editorial expertise in areas we might be lacking and provides training for early career investigators as well as valuable insight into the ``other side'' of the publishing process. Importantly, guest editors only do one paper, so the work does not become a burden of service on them early in their career.

In addition, as a member of the maize genetics steering committee, I have been working all year to revise and establish programs to effectively prevent sexual harassment at the annual meeting (of more than 600 maize geneticsists). I first worked to convince the steering committee to establish a code of conduct for the meeting, and now am pushing to revise traditions regarding alcohol and after hours parties. I have also spend considerable time investigating reporting strategies and options, and have advocated to the committee that we employ the services of a third party company for anoynmous reporting and investigation.

Finally, I have continued service activities for other universities, including promotion letters and serving as an external member of a search committee.

\bibliographystyle{plain}
\bibliography{jri}


\end{document}
