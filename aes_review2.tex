\documentclass[]{article}
\usepackage{fullpage}
\begin{document}

\title{Summary of AES-related activities, since appointment in 2009}
\author{Jeffrey Ross-Ibarra}
\date{}
\maketitle

My program broadly involves using statistical genomic approaches to understand the processes patterning genetic diversity in maize.  This work informs breeding methods, enables the identification of loci underlying agronomic traits, and identifies novel diversity relevant for maize adaptation.  I have worked with public breeders, industry scientists, and academic researchers to apply these approaches to a broad set of elite germplasm, traditional farmer varieties, and wild relatives.  In addition to primary research productivity, this work has provided the opportunity to disseminate our research findings and contribute to the training of future generations of plant scientists.

\section*{Outreach \& Training}
\begin{itemize}
\item Founding and organization of all five years of the UC Davis Plant Breeding Symposium. Negotiated initial and continued funding from DuPont, supervised student committee.
\item Redesigned and deployed UCD Plant Breeding website.
\item Educational presentations (UC Davis Plant Breeding Academy, Illinois Corn Breeders School)
\item Trained postdoctoral scholars who have successfully found research careers in industry (KWS, Monsanto, Dupont Pioneer), government (USDA), an international non-governmental organizations (N2Africa), or public land-grant university (Iowa State)
\item Trained undergraduate students who have gone on to graduate school (including a USDA-funded fellowship) and work in seed and biotech industry.
\end{itemize}

\subsection*{Invited Seminars --- 50 since 2009}
\begin{itemize}
\item Public outreach via social media (twitter, slideshare, figshare) and seminars (e.g. San Francisco Exploratorium)
\item Academic presentations/seminars at universities/conferences both nationally and internationally (France, China, Australia, Mexico, Canada)
\item Presentations to industry stakeholders (Monsanto, Dupont Pioneer, Seminis, BASF, GEM Cooperative)
\end{itemize}

\section*{Research}

\subsection*{Maize Breeding}
\begin{itemize}
\item Identification of individual genes underlying key domestication traits in maize (Studer et al. 2012 Nature Genetics, Wills et al. 2013 PLoS Genetics)
\item Identification of individual genes related to agronomic properties of maize grain, including oil content (Cook et al. 2012 Plant Physiology) and seed weight (Sosso et al. 2015 Nature Genetics
\item Statistical genetic analysis of the role of drift and selection within modern hybrid breeding programs (Gerke et al. 2015)
\item Generation of large genomic and marker resources for breeding, mapping, and genetic analysis (Gore et al. 2009 Science, Chia et al. 2012 Nature Genetics, Bukowski et al. 2015 bioRxiv)
\item Discovery of the important phenotypic contribution of deleterious mutations to maize agronomic traits (Mezmouk et al. 2014 Genes, Genomes, Genetics) and the role of deleterious variants in determining inbreeding success across the genome (Gore et al 2009 Science)
\item Development of a statistical model to predict patterns of genetic variation among smallholder maize farmers (van Heerwaarden 2010 Heredity)
\item Active collaboration with geneticists in Monsanto and Dupont Pioneer to develop methods to identify genomic regions targeted by selection during recent breeding
\item Collaboration with breeders at U. Illinois to investigate the role of deleterious mutations in determining hybrid vigor in elite public maize breeding lines
\end{itemize}

\subsection*{Maize diversity and adaptation}
\begin{itemize}
\item Archaeogenetic analysis of the historical introduction of maize into the United States and genetic makeup of early US maize (de Fonseca et al. 2015 Nature Plants)
\item Reconciliation of the ecological, genetic, and archaeological evidence of the geographic origin of maize domestication (van Heerwaarden PNAS 2011)
\item Identification of loci showing evidence of selection during modern breeding (van Heerwaarden et al. 2012 PNAS), adaptation to novel environments (Takuno et al 2015 Genetics) and domestication (Hufford et al. 2012 Nature Genetics)
\item Documentation of the important role of structural rearrangements in facilitating adaptive change in the maize genome (Fang et al. 2012 Genetics, Pyh\"{a}j\"{a}rvi et al. 2013 Genome Biology and Evolution)
\item Discovery of the important role of gene flow from crop wild relatives in facilitating adaptation to novel environments (Hufford et al. 2013 PLoS Genetics)
\item Assessment of the importance of human cultural differences in impacting maize diversity in Mexico (Orozco-Ramirez et al. 2016 Heredity)
\item Analysis of genetic variation in expression (Swanson-Wagner et al. 2012 PNAS, Makarevitch et al. 2015 PLoS Genetics) and epigenetic changes (Waters et al. 2013 PNAS) among maize lines
\item Evaluation of mating system on diversity in wild populations of maize relatives (Hufford et al. 2011 Molecular Ecology, van Heerwaarden et al. 2010 Molecular Ecology)
\item First demonstration of the mechanism of genetic exchange in plant centromeres (Shi et al. 2010 PLoS Biology)
\end{itemize}

\end{document}