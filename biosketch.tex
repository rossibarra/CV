\documentclass[10pt]{article}
\renewcommand{\rmdefault}{phv} % Arial
\renewcommand{\sfdefault}{phv} % Arial
\usepackage{ucs}
\usepackage[utf8x]{inputenc}
\usepackage{url}
\usepackage[top=2.54cm, bottom=2.54cm, left=2.54cm, right=2.54cm]{geometry}
\usepackage{hyperref}
\usepackage{setspace}
\newcommand{\ignore}[1]{}
\onehalfspacing
\setstretch{1}
\usepackage[explicit]{titlesec}
\usepackage[normalem]{ulem}

\titleformat{\section}{\normalfont}{}{0pt}{\textbf{(\alph{section}) #1}}
%\titleformat{⟨command⟩}[⟨shape⟩]{⟨format⟩}{⟨label⟩}{⟨sep⟩}{⟨before-code⟩}[⟨after-code⟩]
\titleformat{\subsection}
  {\normalfont}{\thesubsection}{1em}{\uline{#1}}
%\usepackage{helvet}
%\renewcommand{\familydefault}{\sfdefault}

%\titleformat{\section}
%  {\normalfont\textbf}{}{}{(\alph{section})\,}
%\renewcommand{\thesection}{(\alph{section})}
%\renewcommand{\thesubsection}{(\uline{subsection})}

\begin{document}

\begin{center}
		\sf\textbf{{Biographical Sketch --- Jeffrey Ross-Ibarra}}
\end{center}
%\begin{minipage}{0.55\linewidth}
%  \href{http://www.plantsciences.ucdavis.edu/plantsciences/}Department of Plant Sciences\\
%  \href{http://www.ucdavis.edu/}{University of California} \\
%  1 Shields Ave.\\
%  Davis, CA 95616
%\end{minipage}
%\begin{minipage}{0.35\linewidth}
%  \begin{tabular}{ll}
%    Phone: & (530) 752-1152 \\
%    Fax: &  (530) 752-4604 \\
%    Email: & \href{mailto:rossibarra@ucdavis.edu}{rossibarra@ucdavis.edu} \\
%    Web: & \href{http://www.rilab.org/}{www.rilab.org} \\
%  \end{tabular}
%\end{minipage}
%\bigskip

%\section{Expertise}

%Dr.\ Ross-Ibarra's research focuses on the population genetics of maize and its wild relatives.  His group uses population genetics to investigate the evolution of natural and cultivated populations, identify loci important for adaptation, and understand the evolutionary forces shaping diversity in the maize genome.
%\bigskip

\section{Professional Preparation}

\begin{tabular}{l l l l}
Institution    \hspace{52mm}              &   Area  \hspace{10mm}     & Degree / Training  \hspace{13mm}    & Dates \\
\hline
University of California Irvine & Genetics & Postdoctoral Research & 2008 \\
University of Georgia & Genetics & PhD & 2006\\
University of California Riverside & Botany & BA, MS & 1998, 2000 \\
\end{tabular}

\section{Professional Appointments}

\begin{tabular}{l l l}
Position & Institution                 & Dates\\
\hline
Professor & University of California Davis &		2016-present \\
Associate Professor & University of California Davis &		2012-2016 \\
Assistant Professor & University of California Davis &		2009-2012 \\
Profesor de Asignatura & Universidad Nacional Aut\'{o}noma de M\'{e}xico & 2001 \\
\end{tabular}

\section{Products}

\subsection*{Most Relevant to the Proposed Research}

\begin{itemize} \setlength{\itemsep}{0pt} \setlength{\parskip}{2pt} \setlength{\parsep}{0pt}


\item \textbf{Gates DJ}$^\S$, Runcie D, Janzen GM, Romero Navarro A,  Willcox M,  Sonder K, Snodgrass SJ, Rodr\'{i}guez-Zapata F,  Sawers RJH, Rub\'{e}n Rell\'{i}n-\'{A}lvarez, Buckler ES, Hearne S, Hufford MB, \textbf{Ross-Ibarra J}$^\S$ (2020). Single-gene resolution of locally adaptive genetic variation in Mexican maize. doi: 10.1101/706739

\item \textbf{O'Brien AM}$^\S$, Sawers RJH, Strauss SY, \textbf{Ross-Ibarra J}$^\S$ (2019). Adaptive phenotypic divergence in teosinte differs across biotic contexts. \textsc{Evolution} 73: 2230--2246

\item \textbf{O'Brien A}$^\S$, Sawers R, \textbf{Ross-Ibarra J}, Strauss  SY$^\S$ (2018) Evolutionary responses to conditionality in species interactions across environmental gradients. \textsc{American Naturalist} 192(6): 715-730.

\item Tiffin P, {\bf Ross-Ibarra J} (2014) Advances and limits of using population genetics to understand local adaptation. \textsc{Trends in Ecology and Evolution} 29:673-680 %Preprint: \url{https://peerj.com/preprints/488/}
%CITES:2471984348452499818

\item {Pyh\"aj\"arvi T}, {Hufford MB}, {Mezmouk S}, {\bf Ross-Ibarra J} (2013) Complex patterns of local adaptation in teosinte. \textsc{Genome Biology and Evolution} 5: 1594-1609.


\end{itemize}

\subsection*{Additional Products}

\begin{itemize} \setlength{\itemsep}{0pt} \setlength{\parskip}{2pt} \setlength{\parsep}{0pt}

\item \textbf{Josephs EM}$^\S$, Berg JJ, \textbf{Ross-Ibarra J}, Coop G (2019) Detecting adaptive differentiation in structured populations with genomic data and common gardens. \textsc{Genetics} 211: 989-1004.

\item \textbf{Lorant A}, \textbf{Ross-Ibarra J}, Maud Tenaillon (2018) Genomics of long- and short- term adaptation in maize and teosinte. \textit{In} \textsc{Statistical Population Genomics},  Dutheil (Ed.), Springer Nature Publishing \textit{In Press}

\item Fang Z, {\bf Pyh\"aj\"arvi T}, Weber AL, Dawe RK, Glaubitz JC, S\'{a}nchez Gonz\'{a}lez J, {\bf Ross-Ibarra C}, Doebley J, Morrell PL$^\S$, {\bf Ross-Ibarra J}$^\S$  (2012) Megabase-scale inversion polymorphism in the wild ancestor of maize. \textsc{Genetics} 191:883-894

\item {\bf Hufford MB}, {\bf Bilinski P}, {\bf Pyh\"aj\"arvi T}, {\bf Ross-Ibarra J}$^\S$ (2012) Teosinte as a model system for population and ecological genomics. \textsc{Trends in Genetics} 12:606-615

\item {\bf Hufford MB}$^\S$, Gepts P, {\bf Ross-Ibarra J} (2011) Influence of cryptic population structure on observed mating patterns in the wild progenitor of maize (\emph{Zea mays} ssp. \emph{parviglumis}).  \textsc{Molecular Ecology} 20: 46-55


\end{itemize}

\section{Synergistic Activities}

\begin{itemize} \setlength{\itemsep}{0pt} \setlength{\parskip}{2pt} \setlength{\parsep}{0pt}
\item Stadler Mid-Career Award in Maize Genetics, 2016
\item Senior Editor, G3
\item Faculty Development Award in recognition of university service, 2015
\item DuPont Young Professor 2012-2014 and faculty advisor DuPont Pioneer graduate student symposium in plant breeding 2012-present
\item Presidential Early Career Award for Scientists and Engineers 2009
\end{itemize}
% \section{Collaborators and Other Affiliations}

% \subsubsection*{Collaborators and Co-editors (Total: 56)}
% \emph{Cornell U} Peter Bradbury, Jeffrey Glaubitz, Susan McCouch, Qi Sun, Feng Tian, Sharon Mitchell;
% \emph{USDA-ARS} Edward Buckler, Sarah Hake, James Holland, Sherry Flint-Garcia, Mike McMullen, Doreen Ware, Jode Edwards;
% \emph{U Southern California} Peter Ralph;
% \emph{UC Davis} Alan Bennet, Daniel Runcie, Ed Taylor, Graham Coop, Keith Bradnam, Ian Korf, David Neale, Am\'elie Gaudin;
% \emph{UC Irvine} Kevin Thornton;
% \emph{Carnegie Institute} Davide Sosso;
% \emph{Stanford} Wolf Frommer;
% \emph{LANGEBIO} Ruairidh Sawers;
% \emph{U Georgia} Kelly Dawe;
% \emph{Arizona State} Reed Cartwright;
% \emph{U Missourri} James Birchler, Katherine Guill, David Wills;
% \emph{Beijing Genomics Institute} Song Chi, Xun Xu;
% \emph{U Wisconsin} John Doebley, Jiming Jiang, Shawn Kaeppler;
% \emph{Syngenta} William Briggs;
% \emph{Monsanto} Lisa Kanizay;
% \emph{Dupont Pioneer} Andy Baumgarten, Justin Gerke, Oscar Smith, Tabare Abadie;
% \emph{U Minnesota} Roman Briskine, Peter Morrell, Chad Myers, Nathan Springer, Peter Tiffin;
% \emph{MIT} Mary Gehring;
% \emph{NC State} Major Goodman;
% \emph{INRA} Clementine Vitte, Maud Tenaillon;
% \emph{Brigham Young} Clinton Whipple;
% \emph{Danforth Center} Anthony Studer;
% \emph{Universidad de Guadalajara} Jesus S\`anchez Gonz\`alez;
% \emph{Iowa State} Carolyn Lawrence;
% \emph{U Hawaii} Gernot Presting;
% \emph{UC Riverside} Mitchell Provance \\

% \subsubsection*{Graduate Advisors and Postdoctoral Sponsors (Total: 3)}
% \emph{UC Riverside} Norman Ellstrand; \emph{U Georgia} James Hamrick; \emph{UC Irvine} Brandon Gaut

% \subsubsection*{Thesis Advisor and Postgraduate Sponsor (Total: 14)}
% { Postdoctoral:} \emph{Iowa State} Matthew Hufford; \emph{Graduate U Advanced Studies} Shohei Takuno; \emph{U Oulu} Tanja Pyh\"aj\"arvi, \emph{KWS} Sofiane Mezmouk; \emph{Wageningen} Joost van Heerwaarden; \emph{USDA} Tim Beissinger; \emph{UC Davis} Kate Crosby, Sayuri Tsukahara, Simon Renny-Byfield, Jinliang Yang { Graduate:} Dianne Velasco, Paul Bilinski, Anna O'Brien, Michelle Stitzer


\end{document}
