\documentclass[10pt]{article}
\renewcommand{\rmdefault}{phv} % Arial
\renewcommand{\sfdefault}{phv} % Arial
\usepackage{ucs}
\usepackage[utf8x]{inputenc}
\usepackage{url}
\usepackage[top=2.54cm, bottom=2.54cm, left=2.54cm, right=2.54cm]{geometry}
\usepackage{hyperref}
\usepackage{setspace}
\newcommand{\ignore}[1]{}
\onehalfspacing
\setstretch{1}
\usepackage[explicit]{titlesec}
\usepackage[normalem]{ulem}

\titleformat{\section}{\normalfont}{}{0pt}{\textbf{(\alph{section}) #1}}
%\titleformat{⟨command⟩}[⟨shape⟩]{⟨format⟩}{⟨label⟩}{⟨sep⟩}{⟨before-code⟩}[⟨after-code⟩]
\titleformat{\subsection}
  {\normalfont}{\thesubsection}{1em}{\uline{#1}}
%\usepackage{helvet}
%\renewcommand{\familydefault}{\sfdefault}

%\titleformat{\section}
%  {\normalfont\textbf}{}{}{(\alph{section})\,}
%\renewcommand{\thesection}{(\alph{section})}
%\renewcommand{\thesubsection}{(\uline{subsection})}

\begin{document}

\begin{center}
		\sf\textbf{{Biographical Sketch --- Jeffrey Ross-Ibarra}}
\end{center}
%\begin{minipage}{0.55\linewidth}
%  \href{http://www.plantsciences.ucdavis.edu/plantsciences/}Department of Plant Sciences\\
%  \href{http://www.ucdavis.edu/}{University of California} \\
%  1 Shields Ave.\\
%  Davis, CA 95616
%\end{minipage}
%\begin{minipage}{0.35\linewidth}
%  \begin{tabular}{ll}
%    Phone: & (530) 752-1152 \\
%    Fax: &  (530) 752-4604 \\
%    Email: & \href{mailto:rossibarra@ucdavis.edu}{rossibarra@ucdavis.edu} \\
%    Web: & \href{http://www.rilab.org/}{www.rilab.org} \\
%  \end{tabular}
%\end{minipage}
%\bigskip

%\section{Expertise}

%Dr.\ Ross-Ibarra's research focuses on the population genetics of maize and its wild relatives.  His group uses population genetics to investigate the evolution of natural and cultivated populations, identify loci important for adaptation, and understand the evolutionary forces shaping diversity in the maize genome.
%\bigskip

% BioSketch:
% The Biographical Sketch should be limited to 2 pages each in length, excluding publications listings. This biographical sketch is required.
% Recommended information includes:
% • Collaborators and Affiliations (for conflicts of interest)
% • Publications and Synergistic Activities

% Save the information in a single .pdf file and attach.

% The biographical sketch should summarize academic and research credentials. This can include earned degrees, teaching experience, employment history, professional activities, honors and awards, and grants received.

% Include a chronological list of all publications in refereed journals during the past 4 years, including those in press. List only those non-refereed technical publications that have relevance to the proposed project. List all authors in the same order as they appear on each paper cited, along with the title and complete reference as these usually appear in journals.



\section{Professional Preparation}

\begin{tabular}{l l l l}
Institution    \hspace{52mm}              &   Area  \hspace{10mm}     & Degree / Training  \hspace{13mm}    & Dates \\
\hline
University of California Irvine & Genetics & Postdoctoral Research & 2008 \\
University of Georgia & Genetics & PhD & 2006\\
University of California Riverside & Botany & BA, MS & 1998, 2000 \\
\end{tabular}

\section{Professional Appointments}

\begin{tabular}{l l l}
Position & Institution                 & Dates\\
\hline
Professor & University of California Davis &		2016-present \\
Associate Professor & University of California Davis &		2012-2016 \\
Assistant Professor & University of California Davis &		2009-2012 \\
Profesor de Asignatura & Universidad Nacional Aut\'{o}noma de M\'{e}xico & 2001 \\
\end{tabular}

\section{Awards}
\begin{itemize}
\item Fellow, AAAS 2021
\item Stadler Mid-Career Excellence in Maize Genetics Award 2016
\item Faculty Development Award in recognition of university service 2015
\item DuPont Young Professor Award 2012
\item Presidential Early Career Award for Scientists and Engineers 2009
\end{itemize}

\section{Mentoring}
I have mentored to date 25 postdoctoral scholars, 13 of whom have continued on in academia (9 as professors) and 10 of whom have found careers in industry. Similarly, among the 6 PhD students I have mentored, 5 remain in academia as postdocs, professors, or ressearch scientists, and 1 has a successful career in industry. I have also mentored a large number of undergraduates, many of whom have continued on to careers in science.

\subsection*{Contributions to Diversity}
\begin{itemize}
  \item Trainer, NSF REU program for research opportunities for diverse undergraduates in evolution and ecology \hfill 2022-2023
  \item Member, IDEA Cmte, Society for Molecular Biology and Evolution \hfill2022-present
  %IDEA workshop, demography, mentoring award for council
  \item Member, pilot program, graduate student mentor training \hfill 2021
  \item Mentor, Graduate Student Mentoring Initiative, Cientifico Latino (2) \hfill 2021 %2 students, multiple meetings each
  \item Advisor, graduate student of color mentoring program (2) \hfill2020-2022 %(2 students, $\sim45$min per week during academic year)  %2 students ~1hr/week total
  \item EVE Diversity cmte (organized 2 workshops)  \hfill 2020-2021
  \item Faculty host, HBCU summer research internship program \hfill 2020
\end{itemize}

\section{Publications (past 4 years)}
\begin{itemize}

  \item Sun S, Wang B, Li C, Xu G, Yang J, Hufford MBH, \textbf{Ross-Ibarra J}, Wang H, Wang L (2023). Unraveling prevalence and effects of deleterious mutations in maize elite lines across decades of modern breeding. \textsc{Mol. Bio. Evol.}. \textit{In Press} 40: msad170
  
  \item  Flint-Garcia S, Hannes Dempewolf H, Feldmann MJ, Morrell PL, \textbf{Ross-Ibarra J} (2023). Diamonds in the Not-So-Rough: Wild Relative Diversity Hidden in Crop Genomes. \textsc{PLoS Biology} 21(7): e3002235

  \item Phillips AR, Seetharam AR, AuBuchon-Elder T, Birchler J, Buckler ES, Gillespie LJ, Hufford MB, Llaca V, Romay MC, Soreng RJ,  Kellogg E, \textbf{Ross-Ibarra J} (2023). A happy accident: a novel turfgrass reference genome. \textsc{G3} 13: jkad073

  \item Hu H, Crow T, Nojoomi S, Schulz, AJ, Hufford MB, Flint-Garcia SF, Sawers RJ, Rell\`{a}n-\`{A}lvarez R, Est\`{e}vez-Palmas JM, \textbf{Ross-Ibarra J}, Runcie DE. Allele-specific expression reveals multiple paths to highland adaptation in maize. \textsc{Mol. Bio. Evol.} 39: msac239
  %CITES: 0
  
  \item \textbf{Rushworth CA},  Wardlaw AM, \textbf{Ross-Ibarra J}, Brandvain Y (2022). Conflict over fertilization underlies the transient evolution of reinforcement. \textsc{PLoS Biology} 20: e3001814 %doi: 10.1101/2020.11.10.377481v1
  %CITES:4305644636578620968
  
  \item Chen L$^*$, Luo J$^*$, Minliang Jin$^*$, \textbf{Yang N$^{*\S}$}, Liu X, Peng Y, Li W, \textbf{Phillips AR}, \textbf{Cameron B}, Bernal J, Rell\'{a}n-\'{A}lvarez R, Saers RJH, Liu Q, Yin Y, Ye X, Yan J, Zhang Q, Zhang X, Wu S, Gui S, Wei W, Wang Y, Luo Y, Jiang C, Deng M, Jin M, Jian L, Yu Y, Zhang M, Yang X, Hufford MB, Fernie AR, Warburton ML, \textbf{Ross-Ibarra J$^\S$}, Yan J$^\S$ (2022). Portrait of a genus: genome sequencing reveals evidence of adaptive variation in \textit{Zea}. \textsc{Nature Genetics} 54: 1736-1745 
  %CITES:14787094829126316446,1364901072355254494,12822792861799271249
  
  \item Li C, Guan H, Jing X, Li Y, Wang B, Li Y-X , Liu X, Zhang D, Liu C, Xie X, Zhao H, Wang Y, Liu J, Zhang P, Hu G, Li G, Li S, Sun D, Wang X, Shi Y, Song Y, Jiao CZ$^\S$, \textbf{Ross-Ibarra J}$^\S$, Li Y$^\S$, Wang T$^\S$, Wang H$^\S$ (2022). Genomic Insights into Historical Improvement of Heterotic Groups during Modern Hybrid Maize Breeding. \textsc{Nature Plants} 8: 750–763
  %CITES:0
  
  \item Guerra-Garcia A, Rojas-Barrera IC, \textbf{Ross-Ibarra J}, Papa R, Pi\~nero D (2022). The genomic signature of wild-to-crop introgression during the domestiation of scarlet runner bean (\textit{Phaseolus coccineus L.}). \textsc{Evolution Letters} 6: 295-307 %doi: 10.1101/2021.02.03.429668v1
  %CITES:8917267558041651296
  
  \item Barnes AC, Rodr\'iguez-Zapata F,Bl\"{o}cher-Ju\'arez KA, \textbf{Gates DJ}, Kur A,  Wang L, Janzen GM,  Jensen S, Est\'evez-Palmas JM, Crow T, Taylor Crow, Aguilar-Rangel R, Demesa-Arevalo E, Skopelitis T, P\'erez-Lim\'on S, Stuttsa WL, Chiu Y-C, Jackson D, Fiehn O, Runcie D, Buckler ES, \textbf{Ross-Ibarra J}, Hufford M, Sawers RJH, Rell\'an-\'Alvarez R (2022). An adaptive teosinte mexicana introgression modulates phosphatidylcholine levels and is associated with maize flowering time \textsc{PNAS} 119: e2100036119  %doi: 10.1101/2021.01.25.426574
  %CITES:16387441939110327044,2541886074262287878
  
  \item \textbf{Horvath R}$^\S$, \textbf{Menon M}, Stitzer M, \textbf{Ross-Ibarra J}$^\S$ (2022). Controlling for Variable Transposition Rate with an Age-Adjusted Site Frequency Spectrum. \textsc{Genome Biology and Evolution}  14: evac016 %doi: 10.1101/2021.08.16.456262
  %CITES:0
  
  \item \textbf{Hudson AI}, \textbf{Odell SG}, Dubreuil P, Tixier M-H, Praud S, Runcie DE, \textbf{Ross-Ibarra J} (2022).  Analysis of genotype by environment interactions in a maize mapping population. \textsc{G3} 12: jkac013 % doi: 10.1101/2021.07.21.453280
  %CITES:6251633562887283614
  
  \item  Samayoa LF, Olukolu  BA, Yang CJ, Chen Q, Stetter MG, York AM, Sanchez-Gonzalez JJ,  Glaubitz JC, Bradbury PJ,  Romay MC, Sun Q, Yang J, \textbf{Ross-Ibarra J}, Buckler ES, Doebley JF, and Holland JB (2022).   Domestication reshaped the genetic basis of inbreeding depression in a maize landrace compared to its wild relative, teosinte. \textsc{PLoS Genetics} 17: e1009797 %doi:10.1101/2021.09.01.458502v1
  %CITES:6632947253215446952,6564788621834623845
  
  \item Perez-Lim\`{o}n S, Li M, Cintora-Martinez GC, Aguilar-Range MR, Salazar-Vidal MN, Gonz\`{a}lez-Segovia E, Blocher-Ju\`{a}rez K, Guerrero-Zavala A, Barrales-Gamez B, Carcano-Macias J,  Nieto-Sotelo J, Martinez de la Vega O, Simpson J, Hufford MB, \textbf{Ross-Ibarra J}, Flint-Garcia S, Diaz-Garcia L, Rell\`{a}n-\`{A}lvarez R, Sawers RJH (2022). A B73 x Palomero Toluque\~{n}o mapping population reveals local adaptation in in Mexican highland maize. \textsc{G3} 12: jkab447 %doi: 10.1101/2021.09.15.460568
  %CITES:4754725764254370698
  
  \item \textbf{Odell SG}, \textbf{Hudson AI}, Praud S, Dubreuil P, Tixier M-H, \textbf{Ross-Ibarra J}, Runcie DE (2022). Modeling allelic diversity of multi-parent mapping populations affects detection of quantitative trait loci. \textsc{G3} 12: jkac011 %doi: 10.1101/2021.07.14.452335
  %CITES:4200777730517717106
  
  \item Calfee E$^\S$, \textbf{Gates DJ}, \textbf{Lorant A}, \textbf{Perkins MT}, Coop GM$^\S$, \textbf{Ross-Ibarra J}$^\S$ (2021). Selective sorting of ancestral introgression in maize and teosinte along an elevational cline. \textsc{PLoS Genetics} 17: e1009810
  %CITES:6213467718649794463
  
  \item \textbf{Stitzer MC}$^\S$, Anderson SN, Springer NM, \textbf{Ross-Ibarra J} (2021). The Genomic Ecosystem of Transposable Elements in Maize. \textsc{PLoS Genetics} 17: e1009768
  %CITES:9872302690859149380
  
  \item Hufford MB, Seetharam AS, Woodhouse MR, Chougule KM, Ou S, Liu J, Ricci WA, Guo T, Olson A, Qiu Y Della Coletta R, \textbf{Tittes S}, \textbf{Hudson AI},  Marand AP, Wei S Lu Z, Wang B, Tello-Ruiz MK, Piri R, Wang N, Kim D, Zeng Y, O'Connor CH, Li X, Gilbert AM, Baggs E, Krasileva KV, Portwood JL, Cannon EKS, Andorf CM, Manchanda N, Snodgrass SJ, Hufnagel DE, Jiang Q, Pedersen S, Syring ML, Kudrna DA, Llaca V, Fengler K, Schmitz RJ, \textbf{Ross-Ibarra J}, Yu J, Gent JI, Hirsch CN, Ware D, Dawe RK (2021). De novo assembly, annotation, and comparative analysis of 26 diverse maize genomes. \textsc{Science} 373:655-662
  %CITES:7617576175906322793
  
  \item Song CB, Wang H, Wu, Y, Rees E, \textbf{Gates DJ}, Burch M,  Bradbury PJ, \textbf{Ross-Ibarra J}, Kellogg EA, Hufford MB, Romay MC, Buckler ES (2021).  Constrained non-coding sequence provides insights into regulatory elements and loss of gene expression in maize. \textsc{Genome Research} gr.266528.120 %doi: 10.1101/2020.07.11.192575
  %CITES:4378530448644595594,13172941186413574025
  
  \item \textbf{Wang L}, Josephs EB, Lee KM, Roberts LM, Rell\'{a}n-\'{A}lvarez R, \textbf{Ross-Ibarra J}$^\S$, Hufford MB$^\S$ (2021). Molecular parallelism underlies convergent highland adaptation of maize landraces. \textsc{Mol. Biol. Evol.} msab119 %doi: 10.1101/2020.07.31.227629
  %CITES:5656251523520837481
  
  \item Muyle A, \textbf{Ross-Ibarra J}, Seymour DK, Gaut BS (2021). Gene body methylation is under selection in \textit{Arabidopsis thaliana}. \textsc{Genetics} 218(2):iyab061 %doi:10.1101/2020.09.04.283333
  %CITES:16295130435365990103
  
  \item Lozano R, Gazave E, dos Santos JPR, Stetter MG, Valluru R, Bandillo N, Fernandes SB, Brown PJ, Shakoor N, Mockler T, Cooper EA, \textbf{Perkins MT}, Buckler ES, \textbf{Ross-Ibarra J}$^\S$, Gore M$^\S$ (2021). Comparative evolutionary analysis and prediction of deleterious mutation patterns between sorghum and maize. \textsc{Nature Plants} 7: 17-24  %  doi: 10.1101/777623
  %CITES:11265103186217917709,1423804503041853700
  
  \item Xu G, Lyu J, Li Q, Liu H, Wang D, Zhang M, Springer NM, \textbf{Ross-Ibarra J}, Yang J (2020). Adaptive evolution of DNA methylation reshaped gene regulation in maize \textsc{Nature Communications} 11: 5539
  %CITES:987817177097090596,14403816878568058683
  
  \item Chen Q, Samayo LF, Yang CJ, Bradbury PJ, Olukolu BA, Neumeyer MA, Romay, MC, Sun Q, \textbf{Lorant A}, Buckler ES, \textbf{Ross-Ibarra J}, Holland JB, Doebley JF (2020).
  The genetic architecture of the maize progenitor, teosinte, and how it was altered during maize domestication \textsc{PLoS Genetics} 16.5:e1008791.
  %CITES:7610480527513064046
  
  \item \textbf{Zeitler L}, \textbf{Ross-Ibarra J}$^\S$, \textbf{Stetter MGS}$^\S$ (2020). Selective loss of diversity in doubled-haploid lines from European maize landraces. \textsc{G3} 10: 2497-2506
  %CITES:40028831341852099,10343727954691350214
  
  \item Wang B, Lin Z, Li X, Zhao Y, Zhao B, Wu G, Ma X, Wang H, Xie Y, Li Q, Song G, Kong D, Zheng Z, Wei H, Shen R, Chen C, Meng Z, Wang T, Li X, Chen Y, Lai J, Hufford MB, \textbf{Ross-Ibarra J}, He H, Wang H (2020). Genome-wide selection and genetic improvement during modern maize breeding. \textsc{Nature Genetics} 52: 565-571
  %CITES:3748829117619888894
  
  \item Torres R$^*$, \textbf{Stetter MG}$^*$, Hernandez R$^\S$, \textbf{Ross-Ibarra J}$^\S$ (2020). The temporal dynamics of background selection in non-equilibrium populations. \textsc{Genetics} 214: 1019-1030
  %CITES:445881558823431528
  
  \item \textbf{Turner-Hissong SD}$^\S$, Mabrey ME, Beissinger TM, \textbf{Ross-Ibarra J}, Pires JC (2020). Evolutionary insights into plant breeding. \textsc{Current Opinion in Plant Biology} 54: 93-100
  %CITES:1066608088401708066
  
  \item  Anderson SN, \textbf{Stitzer MC},  Zhou P, \textbf{Ross-Ibarra J}, Hirsch CD, Springer NM (2019) Dynamic patterns of transcript abundance of transposable element families in maize. \textsc{G3} 9: 3673-3682
  %CITES:13372045702551332348
  
  \item  Anderson SN$^*$, \textbf{Stitzer MC}$^*$,  Brohammer A$^*$, Zhou P, Noshay JM,  O'Connor CH, Hirsch CD, \textbf{Ross-Ibarra J}, Hirsch CN, Springer NM (2019). Transposable elements contribute to dynamic genome content in maize. \textsc{The Plant Journal} 100: 1052-1065
  %CITES:8452519309966999214
  
  \item Wei X,  Anderson SN,  Wang X,  Yang L, Crisp PA,  Li Q,  Noshay J, Albert PS, Birchler JA,  \textbf{Bilinski P}, \textbf{Stitzer MC}, \textbf{Ross-Ibarra J},  Flint-Garcia S,  Chen X,  Springer NM, Doebley JF (2019). Hybrid decay: a transgenerational epigenetic decline in vigor and viability triggered in backcross populations of teosinte with maize. \textsc{Genetics} 213: 143-160
  %CITES:10736570628617082030
  
  \item \textbf{O'Brien AM}$^\S$, Sawers RJH, Strauss SY, \textbf{Ross-Ibarra J}$^\S$ (2019). Adaptive phenotypic divergence in teosinte differs across biotic contexts. \textsc{Evolution} 73: 2230-2246
  %CITES:7213583629655453228
  
  \item  Gonzalez-Segovia E,  P\'erez-Limon S,  C\'intora-Mart\'inez C,  Guerrero-Zavala A,  Jansen G,  Hufford MB, \textbf{Ross-Ibarra J}, Sawers RJH (2019). Characterization of introgression from the teosinte \textit{Zea mays} ssp. \textit{mexicana} to Mexican highland maize. \textsc{PeerJ} 7: e6815. %doi: 10.7287/peerj.preprints.27489v1
  %CITES:8050740196543639739
  
  \item \textbf{Josephs EM}$^\S$, Berg JJ, \textbf{Ross-Ibarra J}, Coop G (2019) Detecting adaptive differentiation in structured populations with genomic data and common gardens. \textsc{Genetics} 211: 989-1004.
  %CITES:4120965274846987962
\end{itemize}




\end{document}
